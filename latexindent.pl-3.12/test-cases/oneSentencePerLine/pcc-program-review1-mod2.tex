% arara: pdflatex: {files: [MathSACpr2014]}
% !arara: indent: {overwrite: yes}
\chapter{Facilities and Support}
\begin{flushright}
	\includegraphics[width=8cm]{xkcd-806-tech_support-punchline}\\
	\url{http://xkcd.com}
\end{flushright}
\section[Space, technology, and equipment]{Describe how classroom space, classroom technology, laboratory space
  and equipment impact student success.}

Over the past few years, efforts by the college to create classrooms containing the same basic equipment has helped tremendously with consistency issues.
The nearly universal presence of classroom podiums with attendant Audio Visual (AV) devices is considerably useful.
For example, many instructors use computer-based calculator emulators when instructing their students on calculator use---this allows explicit keystroking examples to be demonstrated that were not possible before the podiums appeared; the document cameras found in most classrooms are also used by most mathematics instructors.
Having an instructor computer with internet access has been a great help as instructors have access to a wide variety of tools to engage students, as well as a source for quick answers when unusual questions arise.

Several classrooms on the Sylvania campus have Starboards  or Smart Boards integrated with their AV systems.
Many mathematics instructors use these tools as their primary presentation vehicles;  documents can be preloaded into the software and the screens allow instructors to write their work directly onto the document.
Among other things, this makes it easy to save the work into pdf files that can be accessed by students outside of class.
This equipment is not used as much on the other campuses, but there are instructors on other campuses that say they would use them if they were widely available on their campus.

A few instructors have begun creating lessons with LiveScribe technology.
The technology allows the instructor to make an audio/visual record of their lecture without a computer or third person recording device; instructors can post a `live copy' of their actual class lecture online.
The students  do not simply see a static copy of the notes that were written;  the students see the notes emerge as they were being written and they hear the words that were spoken while they were written.
The use of LiveScribe technology is strongly supported by Disability Services, and for that reason alone continued experimentation with its use is strongly encouraged.

Despite all of the improvements that have been made in classrooms over the past few years, there still are some serious issues.

Rooms are assigned randomly, which often leads to mathematics classes being scheduled in rooms that are not appropriate for a math class.
For example, scheduling a math class in a room with individual student desks creates a lot of problems; many instructors have students take notes, refer to their text, and use their calculator all at the same time and there simply is not enough room on the individual desktops to keep all of that material in place.
More significantly,  this furniture is especially ill-suited for group work.
Not only does the movement of desks and sharing of work exacerbate the materials issue (materials frequently falling off the desks), students simply cannot share their work in the efficient way that work can be shared when they are gathered about tables.
It would be helpful if all non-computer-based math classes could be scheduled in rooms with tables.

Another problem relates to an inadequate number of computerized classrooms and insufficient space in many of the existing computerized classroom;  both of these shortages have greatly increased due to Bond-related construction.
Several sections of MTH 243 and MTH 244 (statistics courses), which are normally taught in computerized classrooms, \emph{have} been scheduled in regular classrooms.
Many of the statistics courses that were scheduled in computerized classrooms have been scheduled in rooms that seat only 28, 24, or even 20 students.
When possible, we generally limit our class capacities at 34 or 35.
Needless to say, running multiple sections of classes in rooms well below those capacities creates many problems.
This is especially  problematic for student success, as it hinders students'  ability to register due to undersized classrooms.

Finally, the computerized classrooms could be configured in such a way that maximizes potential for meaningful student engagement and minimizes potential for students to get off course due to internet access.
We believe that all computerized classrooms need to come equipped with software that allows the instructor control of the student computers such as LanSchool Classroom Management Software.
The need for this technology is dire; it will reduce or eliminate students being off task when using computers, and it will allow another avenue to facilitate instruction as the instructor will be able to `see' any student computer and `interact' with any student computer.
It can also be used to solicit student feedback in an anonymous manner.
The gathering of anonymous feedback can frequently provide a better gauge of the general level of understanding than activities such as the traditional showing of hands.

\recommendation[Scheduling]{All mathematics classes should be scheduled in rooms that are either
	computerized (upon request) or have multi-person tables (as opposed to
	individual desks).}

\recommendation[Scheduling, Deans of Instruction, PCC Cabinet]{All computerized classrooms
	should have at least 30, if not 34, individual work
	stations.}

\recommendation[Multimedia Services, Deans of Instruction, PCC Cabinet]{An adequate number of classrooms on all campuses should be equipped with
	Smartboards so that all instructors who want access to the technology can teach
	every one of their classes in rooms equipped with the technology.}

\recommendation[Multimedia Services, TSS]{The disk image for all computerized classrooms should include software that
	allows the podium computer direct access to each student
	computer. }

\section[Library and other outside-the-classroom information
  resources]{Describe how students are using the library or other
  outside-the-classroom information resources.  }
We researched this topic by conducting a stratified sampling method survey of 976 on-campus students and 291 online students; the participants were chosen in a random manner.
We gave scantron surveys to the on-campus students and used SurveyMonkey for the online students.
We found that students are generally knowledgeable about library resources and other outside-the-classroom resources.
The complete survey, together with its results, is given in \vref{app:sec:resourcesurvey}; we have summarized our comments to some of the more interesting questions below.

\begin{enumerate}[label=Q\arabic*.,font=\bf]
	\item Not surprisingly, library resources and other campus-based resources are used more frequently by our on-campus students than by our online students.
	      This could be due to less frequent visits to campus for online students and/or online students already having similar resources available to them via the internet.
	\item We found that nearly 70\% of instructors include resource information in their syllabi.
	      This figure was consistent regardless of the level of the class (DE/transfer level) or the employment status of the instructor (full/part-time).

	      We found that a majority of our instructors are using online resources to connect with students.
	      Online communication between students and instructors is conducted across many platforms such as instructor websites, Desire2Learn, MyPCC, online graphing applications, and online homework platforms.

	      We found that students are using external educational websites such as \href{https://www.khanacademy.org/}{Khan Academy}, \href{http://patrickjmt.com/}{PatrickJMT}, \href{http://www.purplemath.com/}{PurpleMath}, and \href{http://www.youtube.com/}{YouTube}.
	      The data suggest online students use these services more than on-campus students.
	\item The use of online homework (such as WeBWorK,  MyMathLab, MyStatLab, and ALEKS) has grown significantly over the past few years.
	      However, the data suggests that significantly more full-time instructors than part-time instructors are directing their students towards these tools (as either a required or optional component of the course).
	      Additionally, there is a general trend that online homework programs are being used more frequently in online classes than in on-campus classes.
	      Both of these discrepancies may reflect the need to distribute more information to faculty about these software resources.
	\item The Math SAC needs to address whether or not we should be requiring students to use online resources that impose additional costs upon the students and, if so, what would constitute a reasonable cost to the student.
	      To that end, our survey asked if students would be willing to pay up to \$35 to access online homework and other resources.
	      We found that online students were more willing to pay an extra fee than those enrolled in on-campus classes.
	      \setcounter{enumi}{6}
	\item The PCC mathematics website offers a wealth of materials that are frequently accessed by students.
	      These include course-specific supplements, calculator manuals, and the required Calculus I lab manual; all of these materials were written by PCC mathematics faculty.
	      Students may print these materials for free from any PCC computer lab.
	      The website also links to PCC-specific information relevant to mathematics students (such as tutoring resources) as well as outside resources (such as the Texas Instruments website).
	      \setcounter{enumi}{8}
	\item In addition to the previously mentioned resources we also encourage students to use resources offered at PCC such as on-campus Student Learning Centers, online tutoring, Collaborate, and/or Elluminate.
	      A significant number of students registered in on-campus sections are using these resources whereas students enrolled in online sections generally are not.
	      This is not especially surprising since on-campus students are, well, on campus whereas many online students rarely visit a campus.
\end{enumerate}

\recommendation[Math SAC]{The majority of our data suggests that students are using a variety of
	resources to further their knowledge.
	We recommend that instructors continue to educate students about both PCC resources and non-PCC resources.
	We need to uniformly encourage students to use resources such as online tutoring, Student Learning Centers, Collaborate, and/or Elluminate; this includes resource citations in each and every course syllabus.} 

\recommendation[Faculty Department Chairs]{A broader education campaign should be engaged to distribute information to part-time faculty regarding online homework such as WeBWorK, MyMathLab, MyStatLab, and ALEKS.
}

\recommendation[Math SAC]{Instructors should consider quality, accessibility and cost to students when
	requiring specific curriculum materials. }

\section[Clerical, technical, administrative and/or tutoring support]{Provide
  information on clerical, technical, administrative and/or tutoring support.}

The Math SAC has a sizable presence on each of PCC's three campuses and at Southeast Center (soon to be a campus in its own right).
Each campus houses a math department within a division of that campus.
The clerical, technical, administrative, and tutoring support systems are best described on location-by-location basis.

\subsection{Clerical, technical, and administrative support}

Across the district, our SAC has an excellent and very involved administrative liaison, Dr.
Alyson Lighthart.
We would like to thank her for her countless hours of support in attending our SAC meetings and being available to the SAC Co-Chairs.
She provides us with thoughtful feedback and insightful perspectives that help us gather our thoughts and make sound decisions.

\subsubsection{Cascade}
The Cascade math department belongs to the Math, Sciences, Health and PE division.
The math department is located on the third floor of the student services building, sharing a floor with the ROOTS office.
The math department also shares space with allied health support staff, medical professions faculty, medical assisting faculty and the Cascade academic intervention specialists (one of whom is also a math part-time faculty).
Part-time math faculty share 11 cubicles, each with a computer.
Our 7 full-time instructors are paired in offices that open up to the part-time cubicles.
We have space in our offices for another full time faculty member as we lost a temporary full-time position at the start of the 2013 academic year.
In Winter 2014, a collective 42 faculty share one high speed Ricoh printer and one copy machine.
Our division offices are located in another building.
We have a dedicated administrative assistant at the front desk who helps students and faculty most days from 8 {\sc a.m.--5 p.m.} 

\subsubsection{Rock Creek} The Rock Creek math department is located in the same floor as the division it belongs to (Mathematics, Aviation, and Industrial Technology) and it is shared with Computer Science.
Part-time faculty share fourteen cubicles, each with a computer, located in the same office as full-time instructors, that are used to prepare and meet with students.
The sixty-five plus faculty share two high speed printers that can collate, staple and allow double sided printing, and one high speed scanner.
Currently we have reached space capacity and we will have to re-think the current office configuration in order to add one more full-time faculty member next Fall.
Two years ago the Rock Creek math department added a dedicated administrative assistant, which has helped with scheduling needs, coordinating part-time faculty needs, and providing better service to the students.

\subsubsection{Southeast}
The clerical and administrative setup at Southeast has changed, as of Winter 2014.
There was a recent restructuring of divisions.
What used to be the Liberal Arts and Sciences Division split into two divisions: the Liberal Arts and CTE Division (which is in the first floor of Scott Hall, Room 103, where the Liberal Arts and Sciences used to be) and the Math and Science Division (which is on the \nth{2} floor of the new Student Commons Building, Room 214).
All of the math and science faculty are now in this new space, including the part-time instructors (everybody was scattered before, so this is a welcome change).

All of the department chairs have their own offices (with doors), while the rest of the faculty (full-time and part-time) occupy cubicle spaces (approximately 20 cubicles in the space, shared by 4--5 faculty per cubicle).
There are two administrative assistants, one of whom is with the math and science faculty and the other of whom is in charge of the STEM program.
There is also one clerical staff member.

There is one Ricoh printer in the space, along with a fax machine.
Any and all supplies (markers, erasers, etc.
) are located across the hall in a designated
staff room.

\subsubsection{Sylvania}
The Sylvania math department belongs to the Math and Industrial Technology division, which is located in the neighboring automotive building.
The math department is currently located in two separate areas of adjacent buildings as of Fall 2013, when the developmental math faculty officially merged with the math department.
This separation will soon be remedied by construction of the new math department area, scheduled to be completed during Spring 2014.
This new location will be next door to the Engineering department, and will share a conference room, copy machine room, and kitchen.
The math department will include two department chair offices, seventeen full-time instructor cubicles, six additional cubicles shared by part-time faculty, and two flex-space rooms.
Each of the cubicles will have a computer, and there will be two shared laser printers plus one color scanner in the department office.

Our two administrative assistants work an overlapped schedule, which provides dual coverage during the busy midday times and allows the office to remain open to students and visitors for eleven hours.
These assistants do an incredible job serving both student and faculty needs, including:  scheduling assistance, interfacing with technical support regarding office and classroom equipment, maintaining supplies inventory, arranging for substitute instructors, securing signatures and processing department paperwork, guiding students to campus resources, and organizing syllabi and schedules from approximately 70 math instructors.

Our math department has frequent interaction with both Audio-Visual and Technology Solution Services.
Responses by AV to instructor needs in the classroom are extremely prompt--typically within minutes of the notification of a problem.
Since the math department is very technology-oriented, we have many needs that require the assistance of TSS.
Work orders for computer equipment and operational issues that arise on individual faculty computers can take quite a long time to be implemented or to be resolved.
This may be due to the sheer volume of requests that they are processing, but more information during the process, especially notes of any delays, would be welcomed.

\subsection{Tutoring support}
PCC has a Student Learning Center (SLC) on each campus.
It is a testament to PCC's commitment to student success that the four SLCs exist.
However, discrepancies such as unequal distribution of resources, inconsistency in the number and nature of tutors (including faculty `donating time' to the centers), and disparate hours of operation present challenges to students trying to navigate their way through different centers.

\recommendation[PCC Cabinet, Deans of Students, Deans of Instructions, Student Learning Centers]{The college should strive for more
	consistency with its Student Learning Centers.
	We feel that the centers would be an even greater resource if they were more consistent in structure, resource availability, physical space, and faculty support.} 

Over the last five years the general environment of PCC has been greatly impacted by historically-unmatched enrollment growth (see \vref{fig:sec3:DLenrollments,fig:sec3:F2Fenrollments}).
PCC's four Student Learning Centers have been greatly affected by this (see \vref{app:sec:tutoringhours}).
Most notably, the number of students seeking math tutoring has increased dramatically.
Unfortunately, this increase in student need has not been met by increase in tutors or tutoring resources.
As a result the amount of attention an individual student receives has decreased in a substantive way, leaving students often frustrated and without the help they needed.
Consequently, the numbers of students dropped again as students stopped even trying.
While some of this growth has been (or will be) accommodated by increasing the physical space available for tutoring (i.e., by the construction of new facilities at Rock Creek and Southeast), that is still not enough since personnel resources were not increased at the same rate and work-study awards have been decreased significantly.
A comprehensive plan needs to be developed and implemented that will ensure each and every student receives high-quality tutoring in a consistent and consistently accessible manner.

As it now stands, the operation of the SLCs is completely campus driven.
As such, reporting on the current status needs to be done on a campus-by-campus basis.

\subsubsection{Cascade}
Averaging over non-summer terms from Fall 2008 to Spring 2013, the Cascade SLC has served about 680 math students with 3900 individual visits and 8 hours per student per term.
(See \vref{app:tut:tab:SLC} for a full accounting.)

The Cascade SLC has increased its operating hours in response to student demand.
Statistics tutoring is now offered at most times and the introduction of online homework has led to `Hybrid Tutoring', where students receive tutoring while working on their online homework.

At the Cascade Campus, all full-time mathematics instructors and many part-time mathematics instructors volunteer 1--4 hours per week in the SLC to help with student demand.
To help ensure usage throughout the SLC's operational hours, instructors are notified by email of slow-traffic times; this allows the instructors to direct students who need extra help to take advantage of those times.
Other communications such as announcements, ads, and newsletters are sent out regularly.

Full-time faculty have constructed a `First week lecture series' that they conduct on the first Friday of every term (except summer).
It is designed to review basic skills from MTH 20 through MTH 111.
It is run in 50-minute segments throughout the day with a 10-minute break between each segment.
The first offering of this series began in Winter 2012 with 100 students in attendance; the attendance has since  grown steadily and  was up to approximately 300 students by Fall 2013.

The Cascade SLC has formalized both the hiring process and the training process for casual tutors.
The department chairs interview potential tutors, determine which levels they are qualified to tutor, and give guidance as to tutoring strategies and rules.
During their first term, each new tutor is always scheduled in the learning center at the same time as a math instructor, and is encouraged to seek math and tutoring advice from that instructor.

\subsubsection{Rock Creek}
Averaging over non-summer terms from Fall 2008 to Spring 2013, the Rock Creek SLC has served about 690 math students with 3300 individual visits and 10 hours per student per term.
(See \vref{app:tut:tab:SLC} for a full accounting.)

Everyone who works and learns in the Rock Creek SLC is looking forward to moving into the newly-built space in Building 7 by Spring 2014.
The new space will bring the SLC closer to the library and into the same building as the WRC, MC, and TLC.
Students seek tutoring largely in math and science, but increasingly for accounting, computer basics, and also college reading.
Mathematics full-time faculty hold two of the required five office hours at the tutoring center.

Motivated by the high levels of student demand for math tutoring, in 2012/13 the SLC piloted math tutoring by appointment two days per week.
On each of the two days a tutor leads thirty-minute individual sessions or one-hour group tutoring sessions by appointment for most math levels.
After some tweaking of days and times, we have settled on Tuesdays and Wednesdays.
Students who are seeking a longer, more personalized or intensive tutoring session seem to highly appreciate this new service.

Finally, the Rock Creek SLC has benefited over the last three years from collaboration with advisors, counselors, librarians, the WRC, MC, and the Career Resource Center in offering a wide variety of workshops as well as resource fairs to support student learning.

\subsubsection{Southeast}
Averaging over non-summer terms from Fall 2008 to Spring 2013, the Southeast SLC has served about 280 math students with 1200 individual visits and 5 hours per student per term.
(See \vref{app:tut:tab:SLC} for a full accounting.)

The SE SLC staff is looking forward to its move into the new tutoring center facilities when the new buildings are completed.
In the meantime, it has expanded the math tutoring area by moving the writing tutoring to the back room of the tutoring center.

Since the SE Tutoring Center opened in 2004, it has gone from serving an average of 200 students per term (including math and other subjects) to serving an average of 350 students per term in math alone.
With this increase in students seeking assistance, the staff has also grown; the SE SLC now has several faculty members who work part time in the tutoring center.

Many SE math faculty members donate time to the tutoring center.
We have developed a service learning project where calculus students volunteer their time in the tutoring center; this practice has been a great help to students who utilize the tutoring center as well as a great opportunity for calculus students to cement their own mathematical skills.

\subsubsection{Sylvania}
Averaging over non-summer terms from Fall 2008 to Spring 2013, the Sylvania SLC has served about 1100 math students with 6200 individual visits and 7 hours per student per term.
(See \vref{app:tut:tab:SLC} for a full accounting.)

The Sylvania SLC moved into a new location in Fall 2012; it is now in the Library building, together with the Student Computing Center.
The creation of a learning commons is working out well and students are taking advantage of having these different study resources in one place.
Unfortunately, the new SLC has less space available for math tutoring than the prior Student Success Center which has been addressed by restructuring the space.
Since enrollment remains high, having enough space for all students seeking help remains a challenge.

PCC's incredible growth in enrollment created an attendant need for a dramatic increase in the number of tutors available to students.
This increased need has been partially addressed by an increase in the budget set aside for paid tutors as well as a heightened solicitation for volunteer tutors.
Many instructors (both full-time and part-time) have helped by volunteering in the Sylvania SLC; for several years, the center was also able to recruit up to 10 work-study tutors per academic year, but with recent Federal changes to Financial Aid, the Math Center is now only allowed two work-study tutors per year; this restriction has led to a decrease of up to 50 tutoring hours per week.

In addition to tutoring, the Sylvania SLC hosts the self-paced ALC math classes, provides study material, and offers resources and workshops for students to prepare for the Compass placement test.
Efforts are also underway to modernize a vast library of paper-based materials by putting them online and making them available in alternate formats.

\section[Student services]{Provide information on how Advising, Counseling,
  Disability Services and other student services impact students. }

Perhaps more than ever, the Math SAC appreciates and values the role of student services in fostering success for our students.
In our development of an NSF-IUSE proposal (see \vref{over:subsub:nsfiuse}), discussions and planning returned again and again to student services such as advising, placement testing, and counseling.
As we look ahead hopefully toward a realization of the structure that we envisioned, we will keep these services as essential partners in serving our students.
The current status of these services follows.

\subsection{Advising and counseling}
The advising and counseling departments play a vital role in creating pathways for student success; this is especially important when it comes to helping students successfully navigate their mathematics courses.
Historically there have been incidents of miscommunication between various math departments and their campus counterparts in advising, but over the past few years a much more deliberate effort to build strong communication links  between the two has resulted in far fewer of these incidents.

The advising departments have been very responsive to requests made by the mathematics departments and have been clear that there are policies in place that prevent them from implementing some of the changes we would like.

For example, in the past many advisers would make placement decisions based upon criteria that the Math SAC felt weren't sufficient to support the decision.
One example of this was placing students into classes based upon a university's prerequisite structure rather than PCC's prerequisite structure.
When the advisers were made aware that this frequently led to students enrolling in courses for which they were not prepared for success, the advising department instituted an ironclad policy not to give any student permission to register for a course unless there was documented evidence that the student had passed a class that could be transcribed to PCC as the PCC prerequisite for the course.
Any student who wants permission without a satisfied prerequisite or adequate Compass score is now directed to a math faculty chair or to the instructor of the specific section in which the student wishes to enroll.

On the downside, there are things we would like the advisers to do that we have come to learn they cannot do.
For example, for several years the policy of the Math SAC has been that prerequisites that were satisfied at other colleges or universities would only be `automatically' accepted if they were less than three years old.
Many instructors in the math department were under the impression that this policy was in place in the advising department, but it was discovered in 2012 that not only is this policy \emph{not} in place but the policy in fact cannot be enforced by anyone (including math faculty).
Apparently such a policy is enforceable only if explicit prerequisite time-limits are written into the CCOGs.

The advising department had been aware of the prerequisite issue for six or seven years, but somehow the word had not been passed along to the general math faculty.
This serves as an example that both advising supervisors and the math department chairs need to make every effort possible to inform all relevant parties of policy changes in a clear and timely manner.
Towards that end, the math department at Sylvania Campus has now been assigned an official liaison in the Sylvania advising department.
and we believe that similar connections
should be created on the other campuses as well.

With the college's new focus on student completion, the relationship between the math departments and advising departments needs to become much stronger.
Initial placement plays a critical role in completion, as do other things such as enrollment into necessary study skills classes and consecutive term-to-term enrollment through a sequence of courses.
We need to make sure that the advisers have all of the tools necessary to help students make the best choices and the advisers need to help us understand their perspective on the needs of students enrolling in mathematics courses.
To help establish this collaborative environment, a Math SAC ad hoc committee has been formed to investigate and address advising issues, placement issues, and study skills issues;  the committee is going to ask several people involved in advising and counseling to join the committee.
It has been speculated that perhaps such a committee should not be under the direct purview of the Math SAC; if the administration decides to create a similar committee under someone else's direction we ask that any such committee have a large contingent of math faculty.

\recommendation[PCC Cabinet, Deans of Students, Advising]{All four campuses should have an
	official advising liaison and the
	four liaisons should themselves have an established relationship.
	Ideally we would like to have one adviser at each campus dedicated solely to math advising issues.} 

\recommendation[Math SAC, Advising, ROOTS]{ A committee consisting of advisers, math faculty, and other relevant parties (e.g.\ ROOTS representation) should be formed to investigate and establish policies related to student success in mathematics courses.
	The issues to investigate include, but are not limited to,  placement, study skills, and other college success skills as they relate to mathematics courses.} 

\subsection{Testing centers} At the time we wrote our last program review there were very uneven procedures at the various testing centers which caused a lot of problems; the inconsistencies were especially problematic for online instructors and their students---see \cite{mathprogramreview2003}, page 26 .
We are pleased that the testing centers recognized that inconsistency as a problem and they addressed the issue in a forthright way.
The testing centers now have uniform policies and they have made great strides in making their services easily accessible to students and instructors alike.
For example, the ability to make testing arrangements online has been a tremendous help as has the increase in the number of options by which a completed exam can be returned to the instructor.

A limited number of hours of operation remains a problem at each of the testing centers;   evening and weekend hours are not offered and testing times during the remaining time are limited; for example, the Cascade Testing Center offers only four start times for make up exams exam week.
It appears to us that the size of  the facilities and the number of personnel  have not increased in equal parts with the dramatic increase in enrollment.
It also appears that the testing centers have not been given adequate funding to offer hours that accommodate students who can only come to campus during the evening or on a weekend.

This lack of access can be especially problematic for students registered in math courses.
The majority of the math courses at PCC are taught by part-time faculty and these faculty members do not have the same flexibility in their schedule as full-time faculty to proctor their own exams; as such they are especially dependent on the testing centers for make-up testing.
This dependency is all the more problematic since many part-time faculty teach evening or Saturday classes and many of the students in those classes find it difficult to come to campus during `normal business hours.
' Additionally, the
Sylvania math department simply does not have the space required to
administer make-up testing in the office, so 100\% of its faculty are dependent
upon the testing centers for make-up testing;   we realize this puts a strain
on the testing centers.

\recommendation[PCC Cabinet]{We recommend that the space and staffing in the
	testing centers be increased.}

\recommendation[PCC Cabinet, Deans of Students, Testing Centers]{We recommend that make-up
testing be available as late as 9:00 {\sc p.m.}\ at least two days per week,
and that make-up testing hours be available every Saturday.}

As discussed on \cpageref{other:page:disabilityservices,needs:page:disabilityservices}, the Math SAC has a very positive and productive relationship with disability services.
For example, disability services was very responsive when some instructors began to question accommodation requests that contradicted specific evaluation criteria mandated in CCOGs (e.g. testing certain material without student access to a calculator).
Kaela Parks came to the SAC and assured us that any such accommodation request is something an instructor need only consider; i.e., those type of accommodation requests are not mandates on the part of disability services.
The speed with which we received clarity about this issue is indicative of the strong connection that has been forged between the mathematics departments and disability services.

Beginning in the 2012/13 AY, all communication regarding student accommodations (both general and testing-specific) has been done online.
Because of issues such as notifications being filtered to spam files, not all accommodation requests were being read by faculty.
At the  mathematics faculty department chairs' request, Kaela Parks created a spaces page that allow the faculty chairs monitor which instructors have one or more students with accommodation needs and highlights in red any instructor who has an outstanding issue (such as pending exam) that needs immediate attention.
This resource has greatly diminished the number of incidents where a student has an accommodation need that is not addressed in a timely manner.

\section[Patterns of scheduling]{Describe current patterns of scheduling (such
  as modality, class size, duration, DC times, location, or other), address the
  pedagogy of the program/discipline and the needs of students.}
%\section[Patterns of scheduling]{Describe current patterns of scheduling (such as such as modality, class size, duration, times, location, or other).  How do these relate to the pedagogy of the program/discipline and/or the needs of students?} %new version of the question, from Kendra
\label{facilities:sec:scheduling}
The math departments schedule classes that start as early as 7:00 {\sc a.m.}\ and others that run as late as 9:50 {\sc p.m.}  About 80\% of our math classes are offered in a two-day-a week format, meeting either Monday-Wednesday or Tuesday-Thursday.
Some sections are offered in a three-day-a-week format and a few in a four-day-a-week format; sections are offered in these formats to accommodate students who find it helpful to be introduced to less content in any one class session.

We also schedule classes that meet only once a week; some of those classes are scheduled on  Saturdays.
While once-weekly meetings are not an ideal format for teaching mathematics, having such sections creates options for students who cannot attend college more than one day a week.

We offer several courses online, the enrollment in which has jumped dramatically over the past five years (see \vref{fig:sec3:F2Fenrollments} and the discussion surrounding it).
We also offer classes in both a web/TV hybrid and an online/on-campus hybrid format.

On-campus class sizes generally range from 20 to 35 students; that number is typically dependent on the room that is assigned for the class (see \cpageref{needs:page:classsize} and \vref{app:sec:classsize}).
This has led to some inconsistencies among campuses as distribution of classroom capacities is not consistent from one campus to the next.

Teaching online presents unique obstacles for faculty and students alike.
Faculty members, like students, have different methods of addressing these obstacles.
The SAC has a recommended capacity limit of twenty-five for each section on its DL course offerings.
This recommendation was based upon a determination that twenty-five is a reasonable class size given the extra duties associated with teaching online.
Because the attrition rates in online courses can be higher than that in on-campus courses, many DL instructors ask that their class capacity be set at, say, thirty to accommodate for first week attrition.

In addition to increased class sizes that account for anticipated attrition, some faculty members choose to allow additional students when their workload allows for the attendant extra work.
In fact, during Winter 2014, only fifteen out of a total of forty-one DL sections are limited to twenty-five students.
Of the remaining DL sections, seven are capped between twenty-five and thirty, twelve are capped in the mid-to-low thirties, and seven are capped at greater than forty-five.
Further information about scheduling patterns, broken down by campus, can be found in \vref{sec:app:courseschedule}.

There is no specific pedagogical dictates in most of our courses.
Class activities can range from lecture to class-discussion to group-work to student-board-work.
Some instructors provide their students with pre-printed lecture notes and examples, others write notes on the board; some instructors have their students work mostly on computer-based activities, and yet others mostly work problems from the textbook.
The frequency with which each instructor uses each approach is almost entirely up to him/her.
Many instructors have a required online homework component, while others do not.

This diversity of classroom experience has both positive and negative consequences.
On the positive side, it provides an environment that has the potential to address a wide-range of learning styles.
On the negative side, it can lead to very inconsistent experiences for students as they work their way through a sequence.
The inconsistency is probably most prevalent and, unfortunately, most  problematic at the DE level of instruction.
As the Math SAC looks for ways to increase completion rates for students who place into developmental mathematics courses, serious attention will be given to plans that increase the consistency of classroom experience for students; consistency that is built upon evidence-based best practices.

