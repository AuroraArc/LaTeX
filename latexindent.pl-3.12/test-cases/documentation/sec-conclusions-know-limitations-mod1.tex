% arara: pdflatex: {shell: yes, files: [latexindent]}
\section{Conclusions and known limitations}
 \label{sec:knownlimitations}
 There are a number of known limitations of the script, and almost certainly quite a few that are \emph{unknown}!

 The main limitation is to do with the alignment routine discussed on \cpageref{yaml:lookforaligndelims}; for example, consider the file given in \cref{lst:matrix2}.

 \cmhlistingsfromfile{demonstrations/matrix2.tex}{\texttt{matrix2.tex}}{lst:matrix2}

 The default output is given in \cref{lst:matrix2-default}, and it is clear that the alignment routine has not worked as hoped, but it is \emph{expected}.
 \cmhlistingsfromfile{demonstrations/matrix2-default.tex}{\texttt{matrix2.tex} default output}{lst:matrix2-default}

 The reason for the problem is that when \texttt{latexindent.pl} stores its code blocks (see \vref{tab:code-blocks}) it uses replacement tokens.
 The alignment routine is using the \emph{length of the replacement token} in its measuring -- I hope to be able to address this in the future.

 There are other limitations to do with the multicolumn alignment routine (see \vref{lst:tabular2-mod2}); in particular, when working with codeblocks in which multicolumn commands overlap, the algorithm can fail.

 Another limitation is to do with efficiency, particularly when the \texttt{-m} switch is active, as this adds many checks and processes.
 The current implementation relies upon finding and storing \emph{every} code block (see the discussion on \cpageref{page:phases}); it is hoped that, in a future version, only \emph{nested} code blocks will need to be stored in the `packing' phase, and that this will improve the efficiency of the script.

 You can run \texttt{latexindent} on \texttt{.sty}, \texttt{.cls} and any file types that you specify in \lstinline[breaklines=true]!fileExtensionPreference! (see \vref{lst:fileExtensionPreference}); if you find a case in which the script struggles, please feel free to report it at \cite{latexindent-home}, and in the meantime, consider using a \texttt{noIndentBlock} (see \cpageref{lst:noIndentBlockdemo}).

 I hope that this script is useful to some; if you find an example where the script does not behave as you think it should, the best way to contact me is to report an issue on \cite{latexindent-home}; otherwise, feel free to find me on the \url{http://tex.stackexchange.com/users/6621/cmhughes}.
