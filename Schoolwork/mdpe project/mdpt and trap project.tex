\documentclass[letterpaper,12pt]{article}

%% Language and font encodings
\usepackage[english]{babel}

%% page size and margins (changable)
% \setlength{\evensidemargin}{0in}
% \setlength{\oddsidemargin}{0.5in}
% \setlength{\textwidth}{6in}
% \setlength{\topmargin}{0.2in}
% \setlength{\textheight}{8.6in}
% \setlength{\footnotesep}{14pt}
\usepackage[letterpaper,top=2cm,bottom=2cm,left=2cm,right=2cm,marginparwidth=1.75cm]{geometry}
\setlength {\marginparwidth}{2cm}

%% packages
\usepackage{amsmath,amsthm,amssymb,amsfonts}

% theorem types
\newtheorem{theorem}{Theorem}[section]
\newtheorem{lemma}[theorem]{Lemma}
\newtheorem{corollary}[theorem]{Corollary}
\newtheorem{conjecture}[theorem]{Conjecture}
\newtheorem{proposition}[theorem]{Proposition}
\newtheorem*{uconj}{Uniform Boundedness Conjecture}
\theoremstyle{definition}
\newtheorem{definition} [theorem] {Definition}
\newtheorem{claim}[theorem]{Claim}
\newtheorem{example} [theorem] {Example}
\newtheorem{remark} [theorem] {Remark}

\mathchardef\mhyphen="2D
\usepackage{graphicx}
\usepackage{tabularx}
\usepackage{array}
\usepackage[colorinlistoftodos]{todonotes}
\usepackage[colorlinks=true, allcolors=blue]{hyperref}
\usepackage{enumitem}
\usepackage{marginnote}
\usepackage{mathtools}
\usepackage[normalem]{ulem}
\usepackage[utf8]{inputenc}
\usepackage{fancyhdr}
\usepackage{booktabs}
\usepackage{enumitem}
\usepackage{caption}

\usepackage{pgfplots}
\pgfplotsset{compat=1.15}
\usepackage{mathrsfs}
\usepackage{tikz}
\usepackage{tikzit}
\input{default.tikzstyles}

% user-defined macros
\newcommand{\C}{{\mathbb{C}}}
\newcommand{\F}{{\mathbb{F}}}
\newcommand{\N}{{\mathbb{N}}}
\newcommand{\Q}{{\mathbb{Q}}}
\newcommand{\PP}{{\mathbb{P}}}
\newcommand{\R}{{\mathbb{R}}}
\newcommand{\Z}{{\mathbb{Z}}}

\newcommand{\cK}{{\mathcal{K}}}

\newcommand{\fm}{{\mathfrak{m}}}
\newcommand{\fo}{{\mathfrak{o}}}

\newcommand{\Cp}{\C_p}
\newcommand{\Cv}{\C_v}
\newcommand{\Qp}{\Q_p}
\newcommand{\Qpbar}{\bar{\Q}_p}
\newcommand{\Dbar}{\overline{D}}

% user-defined math operators
\DeclareMathOperator{\rad}{rad}
\DeclareMathOperator{\diam}{diam}
\DeclareMathOperator{\divop}{div}

\begin{document}

\begin{titlepage}

 \begin{center}

  \vspace*{\fill}

  \vspace*{0.5cm}

  \huge\bfseries Midpoint and Trapezoidal Project

  \vspace*{0.5cm}

  \large{Henry Yu, Joseph Keeler, Kevin Ma}

  \vspace*{\fill}

 \end{center}

\end{titlepage}

\newpage

\section{Table of Values}

\begin{table}[h]
 \renewcommand*{\arraystretch}{1.8}
 \scalebox{1.1}{
  \hspace{-1cm}\begin{tabular}{|l|l|l|l|l|l|l|}
   \hline
   $f(x)$                     & $[a,b]$ & N    & M              & T             & E                 & $\dfrac{E-T}{E-M}$ \\
   \hline
   $f(x)=2x^{3}-5x^{2}+6$     & $[0,1]$ & $20$ & $4.83375$      & $4.8325$      & $4.8\overline{3}$ & $-1.999976$        \\
   \hline
   $f(x)=3x^{4}-x^{3}+7x$     & $[0,1]$ & $20$ & $3.849063047$  & $3.851874375$ & $3.85$            & $-2.0000500559$    \\
   \hline
   $sin(\frac{\pi x^{2}}{9})$ & $[0,1]$ & $20$ & $0.1152781773$ & $0.115483$    & $0.1153465199$    & $-1.996998944$     \\
   \hline
   $e^{2x}$                   & $[0,1]$ & $20$ & $3.193197384$  & $3.197189713$ & $3.194528049$     & $-2.000251002$     \\
   \hline
   $\sqrt{x+1}$               & $[0,1]$ & $20$ & $1.218966669$  & $1.21892091$  & $1.218951416$     & $-2.0$             \\
   \hline
  \end{tabular}}
\end{table}

\pagebreak

\section{Sides on the Integral}

\raggedright

\begin{large}

 Based on your table, are the midpoint and trapezoid approximations generally on the same side of the exact value of the integral? Justify your answer.

\end{large}

\vspace{1cm}

\begin{minipage}{0.45\textwidth}
 \begin{center}
  Midpoint and Exact
  \vskip 16pt
  \mbox{$4.83375>4.8\overline{3}\hspace{1cm}(M>E)$}
  \vskip 16pt
  \mbox{$3.849063047<3.85\hspace{1cm}(M<E)$}
  \vskip 16pt
  \mbox{$0.1152781773<0.1153465199\hspace{1cm}(M<E)$}
  \vskip 16pt
  \mbox{$3.193197384<3.194528049\hspace{1cm}(M<E)$}
  \vskip 16pt
  \mbox{$1.218966669>1.218951416\hspace{1cm}(M>E)$}
 \end{center}
\end{minipage}
\hfill
\begin{minipage}{0.45\textwidth}
 \begin{center}
  Trapezoid and Exact
  \vskip 16pt
  $4.8325<4.8\overline{3}\hspace{1cm}(T<E)$
  \vskip 16pt
  $3.851874375>3.85\hspace{1cm}(T>E)$
  \vskip 16pt
  $0.115483>0.1153465199\hspace{1cm}(T>E)$
  \vskip 16pt
  $3.197189713>3.194528049\hspace{1cm}(T>E)$
  \vskip 16pt
  $1.21892091<1.218951416\hspace{1cm}(T<E)$
 \end{center}
\end{minipage}

\vspace{1cm}

Based on the information presented, the midpoint and trapezoidal approximations are never on the same side.

\newpage

\section{Approximation Accuracy}

\begin{large}

 Which of the approximations, midpoint or trapezoidal, is generally closer to the exact value of the integral?

\end{large}

\vspace{1cm}

\Large\centerline{$\delta\hspace{0.1cm}(\%)=\dfrac{x_{estimate}-x_{actual}}{x_{actual}}$}
\normalsize

\vspace{0.7cm}
\centerline{Percent Error Formula}

\vspace{1cm}

\begin{minipage}{0.45\textwidth}
 \begin{center}
  Midpoint and Exact
  \vskip 16pt
  $\delta=0.008621\%$
  \vskip 16pt
  $\delta=0.024336\%$
  \vskip 16pt
  $\delta=0.059250\%$
  \vskip 16pt
  $\delta=0.041655\%$
  \vskip 16pt
  $\delta=0.001251\%$
  \vskip 16pt
  $\%\hspace{0.1cm}\text{avg}=0.0270226\%$
 \end{center}
\end{minipage}
\hfill
\begin{minipage}{0.45\textwidth}
 \begin{center}
  Trapezoid and Exact
  \vskip 16pt
  $\delta=0.017241\%$
  \vskip 16pt
  $\delta=0.048685\%$
  \vskip 16pt
  $\delta=0.118322\%$
  \vskip 16pt
  $\delta=0.083319\%$
  \vskip 16pt
  $\delta=0.002503\%$
  \vskip 16pt
  $\%\hspace{0.1cm}\text{avg}=0.054014\%$
 \end{center}
\end{minipage}

\vspace{1cm}

The midpoint approximation and its average is closer to the exact value of the integral than the trapezoidal approximation.

\newpage

\section{Ratios}

\begin{large}

 Interpret the ratios found in the last column of the table. Use the information found to find an expression for E in terms of M and T.

\end{large}

\vspace{1cm}

By looking at the values found in the last column, it is clear that $\dfrac{E-T}{E-M}$ approaches $-2$.\\
By solving for E from the equation,

\vspace{0.7cm}
\centerline{$\dfrac{E-T}{E-M}=-2$}
\vspace{0.7cm}
\centerline{$E-T=-2E+2M$}
\vspace{0.7cm}
\centerline{$3E=T+2M$}
\vspace{0.7cm}
\centerline{$E=\dfrac{T+2M}{3}$.}

\pagebreak

\section{Geometric Argument \# 1}

\begin{large}

 Give a geometric argument to explain why the midpoint approximation gives the exact value of the integral in the case where $f(x)$ has the form $f(x)=-mx+b$.

\end{large}

\vspace{1cm}

Taking the general formula $f(x)=-mx+b$, we find two possible points on the graph. For this example, we take the x values from the x- and y-intercepts as our two points for the endpoints of the integral for the midpoint approximation.

\vspace{1cm}

\centering

\includegraphics[scale=0.25]{desmos-graph.png}

\raggedright

\begin{minipage}{0.45\textwidth}
 \begin{center}
  Midpoint Approximation
 \end{center}
 For n = 1, $\Delta x$
 \vskip 16pt
 $=\ \dfrac{\dfrac{b}{m}-0}{1}$
 \vskip 16pt
 $=\ \dfrac{b}{m}$ .
 \vskip 16pt
 For midpoint approximation, $\Delta x\, [\, f\,(\, \dfrac{x}{2}\, )\, ]$
 \vskip 16pt
 $=\ \dfrac{b}{m}\, \cdot\, \dfrac{b}{2}$
 \vskip 16pt
 $=\ \dfrac{b^2}{2m}$ .
 \vskip 16pt
 The area obtained from the midpoint approximation equals $\dfrac{b^2}{2m}$ .
\end{minipage}
\hfill
\begin{minipage}{0.45\textwidth}
 \begin{center}
  \vspace{-56mm}
  Triangle
 \end{center}
 \vskip 16pt
 Area of triangle\ $=\dfrac{1}{2}\cdot b\cdot \dfrac{b}{m}$
 \vskip 16pt
 $=\ \dfrac{b^2}{2m}$ .
 \vskip 16pt
 The area of the triangle is $\dfrac{b^2}{2m}$ .
\end{minipage}

\pagebreak

\section{Geometric Argument \# 2}

\begin{large}

 Give a geometric argument to explain why the information from your table showed that $M\, <  \int_{0}^{2} e^{x^2} \, dx\, <\, T$ .

\end{large}

\vspace{1cm}

\begin{minipage}{0.45\textwidth}
 \begin{center}
  Trapezoidal Approximation
  \vskip 16pt
  \includegraphics[scale=0.25]{trapezoid.png}
 \end{center}
\end{minipage}
\hfill
\begin{minipage}{0.45\textwidth}
 \begin{center}
  Midpoint Approximation
  \vskip 16pt
  \includegraphics[scale=0.25]{midpoint.png}
 \end{center}
\end{minipage}

\vspace{1cm}

From the two graphs above, it is clear to see that the trapezoidal approximation overestimates the actual area of the function. As for the midpoint approximation, it is a little harder to see that it is an undersestimate. The reason for this is that on the interval [0,2] of the function, it is concave up, leading to these cases being thus.

\pagebreak

\section{Geometric Argument \# 3}

\begin{large}
 Give a geometric argument to explain why the information in your table showed that $T\, <  \int_{1}^{9} \sqrt{x} \, dx\, <\, M$ .
\end{large}

\vspace{1cm}

\begin{minipage}{0.45\textwidth}
 \begin{center}
  Trapezoidal Approximation
  \vskip 16pt
  \includegraphics[scale=0.25]{trapezoid1.png}
 \end{center}
\end{minipage}
\hfill
\begin{minipage}{0.45\textwidth}
 \begin{center}
  Midpoint Approximation
  \vskip 16pt
  \includegraphics[scale=0.25]{midpoint1.png}
 \end{center}
\end{minipage}

\vspace{1cm}

From the two graphs above, it is clear to see that the trapezoidal approximation underestimates the actual area of the function. As for the midpoint approximation, it is a little harder to see that it is an oversestimate. The reason for this is that on the interval [0,2] of the function, it is concave down, leading to these cases being thus.

\pagebreak

\section{Simpson's Rule}

\begin{large}
 Prove that E is Simpson's Rule with a little twist.
\end{large}

\vspace{1cm}

First, recall the formulas for the midpoint rule, the trapezoidal rule, and Simpson's Rule.

\vspace{1cm}

$S_n=\dfrac{\Delta x}{3}\, [f(x_0)+4(x_1)+2f(x_2)+4f(x_3)+2f(x_4)+\ldots+2f(x_{n-2})+4f(x_{n-1})+f(x_n)]$

\vspace{1cm}

$T_n=\dfrac{\Delta x}{2}\{ f(x_0)+2[f(x_1)+f(x_2)+f(x_3)+\ldots +f(x_{n-3})+f(x_{n-2})+f(x_{n-1})]+f(x_n)\}$

\vspace{1cm}

$M_n=\Delta x[f(\dfrac{x_0+x_1}{2})+f(\dfrac{x_1+x_2}{2})+\ldots +f(\dfrac{x_{n-2}+x_{n-1}}{2})+f(\dfrac{x_{n-1}+x_n}{2})]$

\vspace{1cm}

\begin{center}
 \scalebox{2}{\tikzfig{graph1}}
\end{center}

\begin{center}
 Graph showing the relationship between the rules.
\end{center}

\vspace{1cm}

\pagebreak

For Simpson's Rule to work, the number of subintervals (n) must be even.

\vspace{1cm}

\begin{center}
 \scalebox{2}{\tikzfig{graph2}}
\end{center}

\vspace{1cm}

$T_n=\dfrac{2\cdot\Delta x}{2}\{f(x_0)+2[f(x_2)+f(x_4)+f(x_6)+\ldots +f(x_{2n-4})+f(x_{2n-2})]+f(x_{2n})\}$

\vspace{0.5cm}

The trapezoidal rule represents all the even terms of the set 2n.

\vspace{1cm}

$M_n=2\cdot \Delta x[f(x_1)+f(x_3)+f(x_5)+\ldots +f(x_{2n-5})+f(x_{2n-3})+f(x_{2n-1})]$

\vspace{0.5cm}

The midpoint rule represents all the odd terms of the set 2n.

\vspace{0.5cm}

Notice how the trapezoidal and midpoint rule both only have n terms. This is because 2n terms is divided evenly between the two rules.

\vspace{1cm}

$S_{2n}=\dfrac{\Delta x}{3}[f(x_0)+4f(x_1)+2f(x_2)+\cdot +2f(x_{2n-2})+4f(x_{2n-1})+f(x_{2n})]$

\vspace{0.5cm}

Combining the trapezoidal and midpoint rules and substituting the terms inside Simpson's Rule, we obtain,

\vspace{1cm}

$S_{2n}=\dfrac{\Delta x}{3}[f(x_0)+4f(x_1)+2f(x_2)+4f(x_3)+\ldots +4f(x_{2n-3})+2f(x_{2n-2})+4f(x_{2n-1})+f(x_{2n})]$

\vspace{0.5cm}

The even terms are from the trapezoidal rule and also happen to be equal to one another, whereas the odd terms are from the midpoint rule and are four times its amount.

\vspace{1cm}

\pagebreak

$S_{2n}=\dfrac{\Delta x}{3}[\, \dfrac{4M_n}{2\Delta x}+\dfrac{T_n}{\dfrac{2\Delta x}{2}}\, ]$

\vskip 16pt

$=\dfrac{1}{3}\, [\, 2M_n+T_n\, ]$

\vskip 16pt

$=\dfrac{T_n+2M_n}{3}$.

\vspace{1cm}

This is the same formula that was found during problem 4,

\vspace{0.5cm}

\begin{center}
 \hspace{-7mm}$E=\dfrac{T+2M}{3}$.
\end{center}

\vspace{0.5cm}

Comparing to what we have obtained for Simpson's Rule,

\vspace{0.5cm}

\begin{center}
 $=\dfrac{T_n+2M_n}{3}$.
\end{center}

\vskip 16pt

\begin{center}
 \hspace{-18mm}$\dfrac{T+2M}{3}$ = $\dfrac{T_n+2M_n}{3}$
 \vskip 16pt
 \hspace{-27mm}$\therefore\ \ S=E$
\end{center}

\pagebreak

\section*{Contributions}

\vspace{1cm}

\begin{large}

Problems 1-4 was a group effort by all the members in the group. Problem 5 was completed by Henry. Problem 6-7 was completed by Kevin and Joseph. Problem 8 was completed by Henry. The document was compiled by Henry.

\end{large}

\end{document}
