\documentclass[letterpaper,12pt]{article}

%% Language and font encodings
\usepackage[english]{babel}

%% page size and margins
\usepackage[letterpaper,top=2cm,bottom=2cm,left=2cm,right=2cm,marginparwidth=1.75cm]{geometry}
\setlength {\marginparwidth}{2cm}

%% packages
\usepackage{amsmath,amsthm,amssymb,amsfonts}
\mathchardef\mhyphen="2D
\usepackage{graphicx}
\usepackage{tabularx}
\usepackage{array}
\usepackage[colorinlistoftodos]{todonotes}
\usepackage[colorlinks=true, allcolors=blue]{hyperref}
\usepackage{enumitem}
\usepackage{marginnote}
\usepackage{pgfplots}
\pgfplotsset{compat=1.13}
\usepackage{mathtools}
\usepackage[normalem]{ulem}
\usepackage[utf8]{inputenc}
\usepackage{fancyhdr}
\usepackage{booktabs}
\usepackage{enumitem}

\newcommand{\R}{\mathbb{R}}
\newcommand{\N}{\mathbb{N}}
\newcommand{\Z}{\mathbb{Z}}
\providecommand{\C}{\mathbb{C}}

\begin{document}

\begin{titlepage}

 \begin{center}

  \vspace*{\fill}

  \vspace*{0.5cm}

  \huge\bfseries Midpoint and Trapezoidal Project

  \vspace*{0.5cm}

  \large{Henry Yu, Joseph Keeler, Kevin Ma}

  \vspace*{\fill}

 \end{center}

\end{titlepage}

\newpage

\section{Table of Values}

\begin{table}[h]
 \renewcommand*{\arraystretch}{1.8}
 \scalebox{1.1}{
  \hspace{-1cm}\begin{tabular}{|l|l|l|l|l|l|l|}
   \hline
   $f(x)$                     & $[a,b]$ & N    & M              & T             & E                 & $\dfrac{E-T}{E-M}$ \\
   \hline
   $f(x)=2x^{3}-5x^{2}+6$     & $[0,1]$ & $20$ & $4.83375$      & $4.8325$      & $4.8\overline{3}$ & $-1.999976$        \\
   \hline
   $f(x)=3x^{4}-x^{3}+7x$     & $[0,1]$ & $20$ & $3.849063047$  & $3.851874375$ & $3.85$            & $-2.0000500559$    \\
   \hline
   $sin(\frac{\pi x^{2}}{9})$ & $[0,1]$ & $20$ & $0.1152781773$ & $0.115483$    & $0.1153465199$    & $-1.996998944$     \\
   \hline
   $e^{2x}$                   & $[0,1]$ & $20$ & $3.193197384$  & $3.197189713$ & $3.194528049$     & $-2.000251002$     \\
   \hline
   $\sqrt{x+1}$               & $[0,1]$ & $20$ & $1.218966669$  & $1.21892091$  & $1.218951416$     & $-2.0$             \\
   \hline
  \end{tabular}}
\end{table}

\section{Sides on the Integral}

\raggedright

\begin{large}

 Based on your table, are the midpoint and trapezoid approximations generally on the same side of the exact value of the integral? Justify your answer.

\end{large}

\vspace{1cm}

\begin{minipage}{0.45\textwidth}
 \begin{center}
  Midpoint and Exact
  \vskip 16pt
  \mbox{$4.83375>4.8\overline{3}\hspace{1cm}(M>E)$}
  \vskip 16pt
  \mbox{$3.849063047<3.85\hspace{1cm}(M<E)$}
  \vskip 16pt
  \mbox{$0.1152781773<0.1153465199\hspace{1cm}(M<E)$}
  \vskip 16pt
  \mbox{$3.193197384<3.194528049\hspace{1cm}(M<E)$}
  \vskip 16pt
  \mbox{$1.218966669>1.218951416\hspace{1cm}(M>E)$}
 \end{center}
\end{minipage}
\hfill
\begin{minipage}{0.45\textwidth}
 \begin{center}
  Trapezoid and Exact
  \vskip 16pt
  $4.8325<4.8\overline{3}\hspace{1cm}(T<E)$
  \vskip 16pt
  $3.851874375>3.85\hspace{1cm}(T>E)$
  \vskip 16pt
  $0.115483>0.1153465199\hspace{1cm}(T>E)$
  \vskip 16pt
  $3.197189713>3.194528049\hspace{1cm}(T>E)$
  \vskip 16pt
  $1.21892091<1.218951416\hspace{1cm}(T<E)$
 \end{center}
\end{minipage}

\vspace{1cm}

Based on the information presented, the midpoint and trapezoidal approximations are never on the same side.

\newpage

\section{Approximation Accuracy}

\begin{large}

 Which of the approximations, midpoint or trapezoidal, is generally closer to the exact value of the integral?

\end{large}

\vspace{1cm}

\Large\centerline{$\delta\hspace{0.1cm}(\%)=\dfrac{x_{estimate}-x_{actual}}{x_{actual}}$}
\normalsize

\vspace{0.7cm}
\centerline{Percent Error Formula}

\vspace{1cm}

\begin{minipage}{0.45\textwidth}
 \begin{center}
  Midpoint and Exact
  \vskip 16pt
  $\delta=0.008621\%$
  \vskip 16pt
  $\delta=0.024336\%$
  \vskip 16pt
  $\delta=0.059250\%$
  \vskip 16pt
  $\delta=0.041655\%$
  \vskip 16pt
  $\delta=0.001251\%$
  \vskip 16pt
  $\%\hspace{0.1cm}\text{avg}=0.0270226\%$
 \end{center}
\end{minipage}
\hfill
\begin{minipage}{0.45\textwidth}
 \begin{center}
  Trapezoid and Exact
  \vskip 16pt
  $\delta=0.017241\%$
  \vskip 16pt
  $\delta=0.048685\%$
  \vskip 16pt
  $\delta=0.118322\%$
  \vskip 16pt
  $\delta=0.083319\%$
  \vskip 16pt
  $\delta=0.002503\%$
  \vskip 16pt
  $\%\hspace{0.1cm}\text{avg}=0.054014\%$
 \end{center}
\end{minipage}

\vspace{1cm}

The midpoint approximation and its average is closer to the exact value of the integral than the trapezoidal approximation.

\newpage

\section{Ratios}

\begin{large}

Interpret the ratios found in the last column of the table. Use the information found to find an expression for E in terms of M and T.

\end{large}

\vspace{1cm}

By looking at the values found in the last column, it is clear that $\dfrac{E-T}{E-M}$ approaches $-2$.\\
By solving for E from the equation,

\vspace{0.7cm}
\centerline{$\dfrac{E-T}{E-M}=-2$}
\vspace{0.7cm}
\centerline{$E-T=-2E+2M$}
\vspace{0.7cm}
\centerline{$3E=T+2M$}
\vspace{0.7cm}
\centerline{$E=\dfrac{T+2M}{3}$.}

\section{Geometric Argument \#1}

\begin{large}

Give a geometric argument to explain why the midpoint approximation gives the exact value of the integral in the case where $f(x)$ has the form $f(x)=-mx+b$.

\vspace{1cm}



\end{large}

\end{document}
